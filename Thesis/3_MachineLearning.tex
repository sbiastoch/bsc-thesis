\chapter{Machine Learning Grundlagen}
\section{Entscheidungsbäume}
% https://books.google.de/books?id=b3ujBQAAQBAJ&lpg=PR4&ots=sP5pSLEnH2&dq=J.%20R.%20Quinlan%2C%20c4.5&hl=de&pg=PR5#v=onepage&q=J.%20R.%20Quinlan,%20c4.5&f=false
Ross Quinlan C4.5


Attribut Selektion
Entropie der Subsets minimieren \cite[S.~60ff]{Bramer2007}
Entropie neigt dazu Attribute mit vielen Werten zu wählen \cite[S.~72]{Bramer2007}
Gain Ratio in C4.5 benutzt


\section{Inductive Rule Learner}
% http://www.cs.utsa.edu/~bylander/cs6243/cohen95ripper.pdf
Zwei Phasen: Growing und Pruning
Pruning, REduced error Pruning
Growset und Prunset
Solange Prunen wie der Fehler kleiner wird
TOp-down inductive learning
RIPPERk als verbsseurng von IREP
FOIL und TopDownInductiveLearning\cite[S.~701]{RusselNorvig}
regeln werden erstellt durch wiederholte maximierung von FOILs Information Gain Kriterium, in jeder iteration kommt eine bedingung hinzu, bis es keine negativbeispiele mehr gibt
\section{Neural Networks}
Modelieren biologisches Neuron.
Inputs, Outputs, Aktivierungsfunktion
%\section{Evaluationsmaße}
%\subsection{Accurancy}
%\subsection{Precision}
%\subsection{F-Value}
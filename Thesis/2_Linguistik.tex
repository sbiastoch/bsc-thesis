\chapter{Linguistische Grundlagen}

%"linguistische Konzepte"

% Lemma, Flektion, Lexikalisierung, Phonetische Transkription, IPA & DISC 


Zum Verständnis des späteren Vorgehens sind im Folgenden einige Begriffe einzuführen und zu ähnlichen Konzepten von einander abzugrenzen.

\section{Metrische Phonologie}

Metrische Phonologie, Akzentsprache, Tonsprache, Akzentregeln nicht universell \cite[S.~277]{Hall2011}

% Phonetische Umschrift \cite[S.~2ff]{Hall2011}

% Akzentarten \cite[S.~21ff]{Mengel1998}


\subsection{Phonetik vs Phonologie}
Phonetik und Phonologie beschäftigen sich beide mit den Lauten unserer Sprache. Die Phonetik beschäftigt sich dabei mit der praktischen Artikulation von Sprache, ihr Betrachtungsgegenstand sind Schallwellen und deren Wahrnehmung. Die Phonologie hingegen untersucht die zugrunde liegende Struktur von Wörtern und Sprache im allgemeinen. In dieser Arbeit geht es im speziellen um die Metrische Phonologie, welche sich mit den zugrunde liegenden Gesetzen der \textit{Metrik}, also der Betonung beschäftigt. \cite[S. Todo]{Hall2011}

\subsection{Phonem vs Graphem}
Phoneme und Graphme sind sozusagen die \enquote{Atome} von Wörtern, der Stoff, aus dem Wörter gemacht sind. Sie sind die kleinste Informatinseinheiten, welche zur Unterscheidung von Wörtern wichtig sind (distinktive Merkmale). Phoneme beziehen hierbei Töne, Graphem sind Buchstaben.
Manchmal findet man statt Phonem den Begriff Phon. Der Unterschied ist dass Phone lediglich unterschiedliche Laute sind, die nicht notwendigerweise für unsere Sprache bedeutungsunterscheidend sind. So gibt es beispielweise verschiedene Arten von R's, die aber nicht distinktiv im Deutschen sind. Mehrere Phone können somit ein Phonem darstellen. \cite[S.~Todo]{Hall2011}

\subsection{Akzent vs Prosodie}
Generell ist zu unterscheiden zwischen Betonung auf Wort und auf Satzebene. Merkmale auf Satzebene werden auch suprasegmentale Merkmale genannt, Segmente bezeichen hierbei Wortbestandteile, wie z.B. Phoneme oder Silben. Die Betonung einzelner Wörter auf Satzebene ('ICH will das!' vs. 'Ich will DAS!') nennt sich Prosodie und ist unabhängig von der Wortbetonung (ausgenommen stress clash \cite[S. Todo]{Hall2011})
...
Binär
Haupt/Nebenakzent
Hiearchisch
...

\subsection{Sonorität und Phon-Klassen}
Sonoritätshierarchie
Sonoritätsprinzip und Silbifizierung


\subsection{Betrachtungsebenen von Wörtern}
Graphemisch
Phonetisch
CV-Struktur


\section{Machine Learning}

Was ist Machine Learning, was für Arten gibt es, was tu ich davon?

\subsection{Inductive Learning}


\subsection{Decision-Tree Learning}


\subsection{Statistical Learning mit Artificial Neural Networks}

\subsubsection{Perceptron}

\subsubsection{Multilayer Networks}